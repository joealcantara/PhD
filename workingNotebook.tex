%%%%%%%%%%%%%%%%%%%%%%%%%%%%%%%%%%%%%%%%%%%%%%%%%%%%%%%%%%%%%%%%%%%%%%%%%%%%%%%%
%% Document Class
\documentclass[a4paper]{book}
%%%%%%%%%%%%%%%%%%%%%%%%%%%%%%%%%%%%%%%%%%%%%%%%%%%%%%%%%%%%%%%%%%%%%%%%%%%%%%%%
\begin {document}
\section {Things to do}
\begin {itemize}
    \item Learn how to use AWS or Google instances to run deep learning models
\end {itemize}
\section {Ideas to explore}
What if you extract sentences? So you have x list of sentences by controls as data. Each observation is 1 sentence. Each sentence the model reads adds to the probability AD or Not, MCI or Not. \newline
Can you hold meaning / context from previous sentences for repetition? Is this where LSTM's come in? \newline
Me collecting data? What does this look like? Then we can produce probability and confidence? This can then be tackled longditunially. \newline
Does this depend on language too much? Is there sufficient research to support acoustic analysis?
Experiment using this? In theory we migth be able to spot cognitive decline \newline
Read Berisha and Liss \newline
Can I write a paper comparing Reagan, Bush and Trump (use both original Berisha and Liss model and new deep learning model?)
If deficiencies in language are idiosyncratic, how can we generalise? How can we track changes over time?


\section {Bits of writing to go in QR, or Thesis}

\end {document}