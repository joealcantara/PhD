%%%%%%%%%%%%%%%%%%%%%%%%%%%%%%%%%%%%%%%%%%%%%%%%%%%%%%%%%%%%%%%%%%%%%%%%%%%%%%%%
%% Document Class
\documentclass[a4paper]{article}

%% Packages
\input{settings/packages}

%% Page Settings
\input{settings/page}

%% Own Commands
\input{settings/macros}

\title{Literature Review}
\date{2018-04-22}
\author{Jomar Alcantara}

%%%%%%%%%%%%%%%%%%%%%%%%%%%%%%%%%%%%%%%%%%%%%%%%%%%%%%%%%%%%%%%%%%%%%%%%%%%%%%%%
\begin {document}
\maketitle
\newpage
\tableofcontents
\newpage

\section{Introduction to Dementia}

Identification of language impairment is important in Dementia because it aids diagnosis of specific types of dementia, which in turn can alter the prognosis and change the management of the degenerative disorder. As these differences in language are quite subtle, the varying subtypes of dementia are frequently misdiagnosed.

According to a recent report commissioned by the Alzheimer's Society, they estimate the prevalence of Dementia in the UK at approximately 815,000 people. This represents 1 in 14 of the population aged 65 or over. This report also estimates an annual healthcare spend on £4.3 billion of which approximately £85 million is spend solely on diagnosis. They also estimate that the overall impact of dementia (excluding the costs associated with early onset dementia) is £26.3 billion annually.



\section{Assessment of Language function in Dementia}

Language can be defined as the ability to encode ideas into words and/or symbols for the means of communication. Difficulty in language, both spoken and written, are often described as symptoms of various types of Alzheimers' disease (AD). One of the challenges when using speech as a predictor is recognising whether there is a problem with language production or with the motor skills. Impairments in speech that arise from any process that disrupts the neuraxis from the cortex to muscle and encompass dysarthria (disturbance in articulation) and dysphonia (disturbance in the production in vocal sounds). However it is important to use language as a tool to aid the diagnosis of Dementia, as it provides vital clues to aid a clinician in differentiating in the different types of Dementia, which will in turn aid a clinician in an attempt to manage an individual case of Dementia. Due to the subtle nature of the languages changes that are experienced in those who suffer with different types of Dementia, it is often misdiagnosed.



A common way for clinicians' to diagnose dementia is through use of the Mini-Mental State Examination. The Mini-Mental State Examination (MMSE) has shown to be particularly effective at differentiating betweeen Dementia and other psychological disorders such as clinical depression, schitzophrenia and personality disorder and is currently the most widely used measure of diagnosing Dementia within clinical psychology.  However, it is not without it's faults...

\section{Current ways of diagnosing Dementia using Language}

Mini Mental State Examination (MMSE)
Addenbrooke's Cognitive Examination (ACE-III)
Dementia Rating Scale (DRS)
Western Aphasia Battery (WAB)
Boston Diagnostic Aphasia Examination (BDAE)
Boston Naming Test (BNT)

\section{Existing studies that have used Natural Language Processing to help in diagnosing psychological difficulties}

Psychology - Schitzophrenia

\section{What progress has been made in this area already?}

\section{State of literature into Natural Language processing techniques}

\section{Conclusion}
Given the burden on the diagnosis of dementia on clinicians, it appears to be useful to find some non-invasive protocols for the early diagnosis of dementia. 
\end {document}