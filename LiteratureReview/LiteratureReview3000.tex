%%%%%%%%%%%%%%%%%%%%%%%%%%%%%%%%%%%%%%%%%%%%%%%%%%%%%%%%%%%%%%%%%%%%%%%%%%%%%%%%
%% Document Class
\documentclass[a4paper]{article}

%% Packages
\input{settings/packages}

%% Page Settings
\input{settings/page}

%% Own Commands
\input{settings/macros}

\title{Literature Review}
\date{2018-07-13}
\author{Jomar Alcantara}

%%%%%%%%%%%%%%%%%%%%%%%%%%%%%%%%%%%%%%%%%%%%%%%%%%%%%%%%%%%%%%%%%%%%%%%%%%%%%%%%
\begin {document}
\maketitle
\newpage
\tableofcontents
\newpage
\section{Introduction}
\par
According to a recent report commissioned by the Alzheimer's Society in 2015, they estimate the prevalence of Dementia in the UK at approximately 815,000 people. This represents 1 in 14 of the population aged 65 or over and 1 in 79 in the general population. This report also estimates an annual healthcare spend on £4.3 billion of which approximately £85 million is spend solely on diagnosis. They also estimate that the overall impact of dementia (excluding the costs associated with early onset dementia) is £26.3 billion annually. Globally, this picture is a lot bleaker. A recent report suggests that in 2015 there were 46 million people with a diagnosis of dementia and that number is expected to hit 131.5 million by 2050 \cite{Prince2015}. The report also states that the worldwide cost of dementia in 2018 is estimated to be in the region of one trillon US dollars. \newline
\par
Those who are diagnosed with dementia suffer from a number of cogniive, emotional and behavioural problems. In terms of cognitive problems, they suffer from deficiencies to memory, planning and organising, speech and language, visuo-spatial tasks and difficulties in orienting in time and place. From an emotional perspective, those with dementia can suffer from mental health problems such as depression, anxiety and irritability.  \newline
\par
According to the DSM 5 \cite{AmericanPsychiatricAssociation2013}, those with Mild dementia suffer from noticeable word-find difficulty. They may substitute general terms for more specific terms. They may avoid the use of specific names of acquaintances. There may be grammatical errors involving subtle omission or incorrect use of articles, prepositions, auxiliary verbs, etc. Those who have progressed from Mild to Major depression also have difficulties with expressive or receptive language. They will often use general-use phrases such as "that thing" and "you know what I mean," and prefers general pronouns rather than names. With severe impairment, sufferers may not even recall names of closer friends and family. Idiosyncratic word usage, grammatical errors, and spontaneity of output and economy of utterances occur. Stereotypy of speech occurs; echolalia and automatic speech typically precede mutism.\newline
\par
% In the paragraph below, cite Goodglass and Kaplan (1983) and/or Goodglass (2000)
In assessing whether a client has AD, part of a clinicans job is to assess langugage capability. As described above, AD has a significant impact on language and a clinician has access to tools designed to test lanaguage capabilities. An example of this is part of the Boston Diagnostic Aphasia Examination (BDAE), called the Boston Cookie Theft picture description task \cite{} . In this task participants are asked to describe a picture presented to them in as much detail as possible. The picture itself depicts a familiar domestic scene and would not require participants to use any vocabulary beyond that learned in childhood. It was originally designed to assess Aphasia, but has shown itself to be useful in the diagnosis of AD as well \cite{GilesPattersonHodges1995}\newline
\par
The study of language decline in Dementia is not a new field of study. The focus of this work has been in understanding in which areas of language are impacted by dementia. Emery \cite{Emery2000} completed a literature review in this area and structured her review into four levels of language. There are four levels of language according to this perspective: Phonology, Morphology, Syntax and Semantics and her review looked at the idea that language and the processes involved in language are hierarchical. She proposes that language goes from simple units of construction, and build layers of complexity and sophistication. She found that people with AD generally had intact Phonology and Morphology but more impaired Syntax and Semantics. Emery states that you can see language decline as Hierarchical. According to a semiotic model there are four ranks of language based on their cognitive complexity. These ranks are Phonology, Morphology, Syntax and Semantics. She conducted a review studies which looked at language decline from the perspective of these hierarchical ranks. Her conclusion is that language decline is related to the complexity of the language task given to a participant and that language decline is hierarchical in that the language forms we learn last (the most complex language forms) are the first to deteriorate.\newline
\par
A lot of research since Emery's literature review \cite{Emery2000} have looked supported her ideas. \newline
\par
It is clear from both the clinical diagnostic criteria and supporting research that language is impacted in those with AD. However, one of the costs of analysing in this way is the idea that there is a huge burden on trained practitioners, be it clinical psychologists, audio transcribers and text encoders, to facilitate the process of collecting data and analysis. The field of machine learning and natural language processing could be used to improve the accuracy and lessen the human cost of this research as well as provide new insights into the difficulties that AD suffer in terms of language decline.  
The purpose of this review is to seek to understand what developments have been made in the use of Machine Learning and Natural Language processing to aid the diagnosis of Mild Cognitive Impairment (MCI) and Alzheimer's Disease (AD). A search of the literature was conducted mainly from a psychological perspective using ProQuest (PsychArticles), SCOPUS, Web of Science. The following results were found (Table 1). All papers were the reviewed for relevance by reading the abstract and full text where appropriate. Papers were included where there was an attempt to classify participants as MCI, AD or Healthy using speech and/or language by using Machine Learning techniques. We excluded any papers that focused on other forms of dementia or cognitive impairment, as well as any papers in which the language being analysed was not English. This resulted in 21 journal articles and/or conference papers which form this review.

\begin{center}
	\begin{tabular}{ | l | l | l | p{3cm} |}
		\hline
		Database & Number of Results & Search Terms  \\ \hline
		ProQuest(PsychArticles) & 1484 Results & Language AND Decline AND Dementia \\ \hline
		ProQuest(PsychArticles) & 486 Results  & Language AND Decline AND Dementia AND Speech \\ \hline
		ProQuest(PsychArticles) & 159 Results & Machine Learning AND Dementia AND Language \\ \hline
		Web of Science & 1207 Results  & Language AND Decline AND Dementia   \\ \hline
		Web of Science & 151 Results  & Language AND Decline AND Dementia AND Speech  \\ \hline
		Web of Science & 34 Results & Machine Learning AND Dementia AND Language \\ \hline
		Scopus & 791 Results & Language AND Decline AND Dementia  \\ \hline
		Scopus & 91 Results & Language AND Decline AND Dementia AND Speech   \\ \hline
		Scopus & 29 Results & Machine Learning AND Dementia AND Langauge \\ \hline
		
	\end{tabular}
\end{center}

\section{Connected Speech}
\par 
There is an increasing body of literature which supports the use of machine learning to analyse speech to evidence cognitive decline. Orimaye et al (2017) investigate the use of machine learning algorithms to detect differences in syntactic, lexical and n-gram linguistic biomarkers to distinguish between those with probably AD and healthy controls, The authors found significant differences in the uses of all three types of biomarkers for those with AD and healthy controls. Their results show that the top 1000 n-gram features plus the twenty-three syntactic and lexical features was the most successful at differentiating the two groups (0.93 AUC)). However, there were some limitations of the study. The context of the audio content was very specific, limited to one specific description task, and therefore using general speech as proposed in this project would potentially use a different set of linguistic features which may or may not have similar predictive power. In addition, the transcriptions were encoded using the CHAT format which is a framework for manually annotating speech and the challenge would be finding a way to automate this. This paper is noteworthy as it provides evidence that using machine learning to identify participants with MCI is possible and it provides some ideas of potential features which can be used. Unique to this paper is the use of n-gram features, which could be explored in the proposed project.\newline
\par
Asgari, Kaye and Dodge (2017) also looked at the linguistic characteristics of older adults with mild cognitive impairment (MCI) vs healthy controls where they hypothesised that they would be able to predict those with MCI, a distinguishing characteristic of Alzheimer's disease and other variations of dementia. Using recordings of unstructured conversations (with standardized preselected topics across subjects) between interviewers and interviewees they grouped spoken words using Linguistic Inquiry and Word Count (LIWC) which is a technique used to categorize words into features such as negative and positive words. They then applied support vector machines (SVM's) and random forest classification algorithms to investigate whether machine learning could be used to distinguish between those with MCI and healthy controls. They were able to successfully used machine learning algorithms to distinguish between these two groups with an accuracy of 84. The authors report that this method is highly reliant on high-fidelity transcription of the conversations which is labour intensive, but they anticipate that technology is advancing sufficiently quickly that automated high-quality transcriptions are possible in the near future. This paper provides a different perspective in how to tackle the problem deriving language features from speech using a different framework. The challenges both Orimaye et al and these authors experienced were around the automation of transcription and this would potentially be an area to explore in the proposed project. Asgari et al (2017) used the Linguistic Inquiry and Word Count (LIWC) methodology, developed by Pennebaker et al (2007) to generate features for their paper. Pennebaker, Boyd, Jordan and Blackburn (2015) have since developed their framework. The LIWC is designed to categorize and evaluate the various emotional, cognitive and structural components which are present within samples of speech and/or written text. This method, Linquistic Inquiry and Word Count (LIWC) has evolved from it's initial incarnation in 1993 to it's last update in 2015. The premise is the use of words from particular categories provide an insight into the psychological processes and/or diagnoses an individual has. The latest version of this methodology uses a dictionary of 6400 words, word stems and select emoticons and assigns these constructs to various categories. This has been updated to include modern uses of language such as 'text speak'. This manual goes summarises the process of constructing the framework as well as discussing the reliability and validity of the measure and a review of studies using the LIWC was conducted by Tausczik and Pennebaker (2010) support the notion that the LIWC is valid across multiple psychological domains. This paper is useful as it provides a useful and validated framework from which the proposed project can potentially derive language features.\newline
\par
Currently formal diagnosis of depression is expensive and time-consuming involving both the use of questionnaires and the use of a trained professional to assess an individual. Schwartz et al, explored the use of language as an aid to diagnose depression in a naturalistic setting i.e. facebook status updates. The authors used both the LIWC (described above) and n-grams as features to distinguish those with depression from healthy individuals and were able to track participants levels' of depression through language successfully.Whilst the authors were able to use a reasonably sound method of measure degree of depression and have evidenced a methodology that is able to track a levels of depression over time, it would be more helpful to use a more clinically relevant measure of depression and anxiety i.e., PHQ-9 (Kroenke, 2001) and GAD-7 (Spitzer et al, 2006) to add validity to their findings. This paper is interesting as it uses both LIWC and n-grams as a method of deriving features from text / speech for depression which, as described above, have also been used as features to distinguish MCI from healthy participants. Given the comorbity of depression and dementia (Meyers, 1998), it would be interesting for the proposed project to further analyse the link between dementia and clinical depression via the use of language. This paper also supports the idea that language changes over time and is a marker for deterioration in both those with depression and dementia. \newline
\par
Fraser, Meltzer and Rudzicz (2015) looked at connected speech using the DementiaBank corpus. They found that there were four factors which informed the classification of participants as either healthy or AD. These four factors were semantic impairment, acoustic abnormality, syntactic impairment and information impairment.\newline
\par
\section{Conclusion}
\par
Identification of language impairment is important in Dementia because it aids diagnosis of specific types of dementia, which in turn can alter the prognosis and change the management of the degenerative disorder. As these differences in language are quite subtle, the varying subtypes of dementia are frequently misdiagnosed.
\newline
\par 
Given the burden on the diagnosis of dementia on clinicians, it appears to be useful to find some non-invasive protocols for the early diagnosis of dementia. It has already been shown that analysis of speech and language has shown markers that pre-date the official diagnosis of dementia (Snowdon et al, 1996;)\cite{Berisha2015}. A significant amount of research has gone into the use of machine learning techniques to potentially look at the automated classification of participants with MCI and/or AD, however this is a new area of research and there are some gaps in our knowledge. Firstly, the vast majority of the research described above looks at pre-existing datasets like the DementiaBank corpus (CITE). Whilst this is useful for 'backtesting the data', it would be useful for this process to be tested live to see whether these results can be replicated now. \newline 
\par
A lot of the research was conducted using the idea of Picture Description, and whilst this is a valid measure of testing the linguistic capability of a participant, there are some flaws with this methodology. The picture description task 
\bibliographystyle{unsrt}
\bibliography{LiteratureReview3000}
\end {document}