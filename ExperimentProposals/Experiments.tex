\documentclass{article}

\title{Comparing the presidential news conferences of Ronald Reagan, George H.W. Bush and Donald Trump - A exploration of deep learning for the purposes of tracking language changes over time.\newline \\\large Experimental Proposal 1 }
\date{2018-07-19}
\author{Jomar Alcantara}

\begin{document}
\maketitle
\newpage
\tableofcontents
\newpage
\section{Abstract}
\section{Introduction}

Berisha and Liss compared the press conferences of Ronald Reagan (RR) and George Herbert Walker Bush (GHWB) and found that there were significant markers found that were able to distinguish between RR and GHWB in terms of cognitive impairment. The rationale for comparing these two presidents was that they were the closest in age to each other. With the appointment of Donald Trump (DT) as president of the United States, this affords us an opportunity to compare and contrast Berisha and Liss's results with data from a new and arguably more comparible subject in DT.\newline
\par 
One of the criticisms of the current literature is the amount of approaches that are taken in the pursuit of features / attributes which can be used to predict cognitive impairment. One of the key dependencies that traditional machine learning algorithms have is on the features that are given to the algorithm. Therefore, in some sense, the success or failure of a traditional machine learning algorithm lies in part on the quality of the features given to it. 
\subsection{Hypotheses}
Our aim is to look at the press conferences of RR, GHWB and DT. These are widely available, unedited, on a website (CITE). These press conferences consists of two parts. The first part is a prepared statement which is usually, in part, at least co-written by a member of White House staff, typically a speech writer. The second part is a Question and Answer session, in which the president takes questions from members of the press. It is this part of the press conference in which we shall focus.
\subsubsection{Hypothesis 1}
We expect to find that both traditional forms of machine learning algorithms and deep learning algorithms are able to differentiate between those with cognitive impairment and healthy controls. 
\subsubsection{Hypothesis 2}
We expect to replicate the results of Berisha and Liss.
\subsubsection{Hypothesis 3}
We expect there to 
\section{Methodology}
\section{Results}
\section{Discussion and Future Work}
\section{Conclusion}
\bibliographystyle{unsrt}
\bibliography{ExperimentReport}
\end{document}