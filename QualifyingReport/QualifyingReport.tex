\documentclass{article}

\title{Qualifying Report}
\date{2018-05-22}
\author{Jomar Alcantara}

\begin{document}
	\maketitle
	\newpage
	\section{Overview of Problem}
	Dementia has been identified as one of those fast growing difficulties facing the world. A recent report suggests that in 2015 there were 46 million people with a diagnosis of dementia and that number is expected to hit 131.5 million by 2050(CITE). The report also states that the worldwide cost of dementia in 2018 is estimated to be in the region of one trillion US dollars.
	\newline
	\par
	In 2009, the Department of Health in it's National Dementia strategy made 'early diagnosis and support' one of it's key themes as part of this strategy. A lot of work, therefore, has gone into trying to find ways of improving the early diagnosis of Alzheimer's Disease (AD) and Mild Cogntive Impairment (MCI). As described above, the numbers of those suffering from AD and MCI are going to increase as the population ages and thus it is important that we utilise technology wherever possible to aid clinicians. At the present time diagnosis is typically conducted at memory clinics by trained clinicians. I theorise that we might be able to use diagnose patients in their own homes utilising smart-home technology. 
	\newline
	\par
	Language analysis is an important part of assessing those suspected of suffering with AD or MCI and this language analysis in part informs an official clinical diagnosis. This analysis often takes the form of tasks such as Free Cued Selective Recall Task, the Boston Cognitive Assessment, and the Mini Mental State Examination which are all quite targeted in the language required of the person being assessed. 
	\newline
	\par
	A step towards the ambient detection of AD and/or MCI in population requires that we look for similar features in language used naturally as part of every day conversation. Part of this work therefore is to see whether spontaneous discourse, such a semi-structured interview is a task that has the ability to put pressure both the cognitive and linguistic systems as it requires the ability to understand speech, formulating a coherent answer and then responding in an appropriate way. Berisha et al \cite{Berisha2015}, has shown through language analysis in this area that there are marked differences in this process between those who have a diagnosis of AD and healthy controls. The next part of this work looks at how technology, particularly the use of smart home devices such as the Amazon Echo and Google Home families of devices can be used to capture data that can then be analysed.
	\newline
	\par
	The potential impact of this research is immense. Research has shown that early diagnosis of people with AD or MCI improves sufferers quality of life and can slow the progress of the disease. Equally, early diagnosis can increase the number of research opportunities for understanding the early stages of dementia and how the disease progresses so that more research can be conducted which may, in the future, lead to new treatments and other interventions.
	
	\section{Literature Review}
	
	\section{Work already carried out}
	
	\subsection{Literature Review - April 2018 to July 2018}
	A literature review was carried out from several perspectives in order to understand both an applied and theoretical perspective. Thus a search for papers was conducted in both the psychology and computer science domains, as well as papers which intersected these areas and were relevant to this particular problem.
	\newline
	SEARCH - ProQuest (PsychArticles) - Search Terms (Language AND Decline AND DEMENTIA) - 1484 Results \newline
	SEARCH - ProQuest (PsychArticles) - Search Terms (Language AND Decline AND DEMENTIA AND Speech) - 486 Results \newline
	SEARCH - Web Of Science - Search Terms (Language AND Decline AND DEMENTIA AND Speech) - 151 Results\newline
	SEARCH - Web of Science - Search Terms (Language AND Decline AND DEMENTIA) - 1207 Results\newline
	SEARCH - Scopus - Search Terms (Language AND Decline AND DEMENTIA AND Speech) - 91 Results\newline 
	SEARCH - Scopus - Search Terms (Languge AND DEMENTIA AND Decline) - 791 Results
	\newline
	Studies that met the following criteria were included: articles that addressed the association between language and all forms of dementia. Studies with one or more of the following characteristics were not included in this literature review: articles and dissertations that did not meet the selected inclusion in different databases and those that were not available in it's original form. 	
		
	\subsection{Skills development - April 2018 to March 2019}
	\textbf{IS-4001 Research Skills and Professional Development Module - Aston University - 3rd May, 10th May, 16th May - 45 Hours} \newline
	\par
	Day 1 consisted of an introduction to the module (IS4001), a rough guide to how to start a PhD and a look at how to conduct and write a literature review. This looked at the concept of a PhD and the meaning of original research. This also talked about the supervisor relationship and understanding the role of the supervisor in the PhD. In addition, we explored the idea of our expectations around supervision. We then went on to look at what skills, qualities and attitudes are necessary to be a good researcher including the idea of what consitutes ethical research and finally we looked at the obstacles to progress on a PhD and what we can do as students. In the afternoon, we looked at how a way of conducting a literature review. This talked about what a literature review is, along with the different types of literature review (Narrative, Systematic and Meta-Analysis). We talked about the function of a literature review, and how this fits in with the writing up of a thesis. We then explored the stages that the literature review process contains, from defining the scope of the research, to coming up with a plan on how to collect sources of information to synthesising and reporting the findings of the review. This part of the lecture looked in detail at each of these areas and included some examples to put the theory of the literature into context. Finally, we talked about the the process of a literature review in the context of the written assignment. 
	\newline
	\par
	Day 2 consisted of a presentation on Effective Presentation skills, what consitutes a perfect poster presentation, research project planning and management and how to work effectively in research teams. In the presentation on effective presentation skills we talked about different ways in which we could plan a presentation and some of the key concepts in terms of communication and engaging the audience. We also talked about different presentation styles. This was with a view to thinking about our presentation assessment. We then went on to talk about what constitutes a poster presentation, and what the general features and conventions of poster presentations were. We then went through some examples of what made poster presentations good and bad. In the afternoon we went through how to plan and manage a research project effectively. Our goals were to see our research projects as a series of tasks which needed to be completed in a sequence, and to be able to plan this in a way that makes good use of time and energy as well a mechanism for keeping track of progress. We used the Association of Project Managers (APM) protocol for managing a project and we looked at the notion of having a Gantt chart to have a graphical representation of research project. We then talked about these processes in relation to the research plan which needs to be produced for assessment. Finally we talked about how to work and communicate effectively in teams. We talked about how to work effectively within a team, and specified the difference between groups and teams. We talked about the different roles that people have within teams, and the idea that all teams have individuals that take on these distinct roles. We talked about how to identify our own-strengths within and how we might facilitate good teamwork withing a research team using our strengths. Finally we talked about some of the the results of poor communication within a team.
	\newline
	\par
	The assessments for this module were a 3,000 word literature review with a focus on process rather than content (See Appendix A), a ten minute presentation based on my research project (See Appendix B) and the submission of a project plan spanning the first year of my research(See Appendix C).  
	\newline
	\par
	\textbf{Alexa Skills Training Day - Amazon - May 9th 2018 - 7 Hours}
	\newline
	\par
	Test
	\newline
	\textbf{Deep Learning Online Course - deeplearning.ai - Dates - 70 Hours}
	Summary Here

	\subsection{Lab Book}
	\subsection{Publication plan}
	\textbf{Year One}
	\textbf{Year Two}
	\textbf{Year Three}
	
	\section{Proposed Experiment}	
	
	
	\section{Proposed Timetable}
	

	\subsection{January 2019}
	Friday 11th January - Rough Draft of Qualifying Due
	Friday 18th January - Complete first rough pass of capturing and processing speech.
	
	\subsection{February 2019}
	Friday, 1st February - Qualifying Report Due
	\subsection{April 2019}
	Monday, 1st April - Viva Due by this point.
	
	\section{Reflection on my experience so far}
	
	\section{Provisional Table of Contents}
	\subsection{Chapter 1: Introduction}
	\subsection{Chapter 2: Literature Review}
	\subsection{Chapter 3: New theories, Description of experiments}
	\subsection{Chapter 4: Results}
	\subsection{Chapter 5: Discussion}
	\subsection{Chapter 6: Conclusion}
	
	%Complete appendices for each of the assessment. How do you do appendices in LaTeX
	
	\bibliographystyle{unsrt}
	\bibliography{QualifyingReport}

	\appendix
	\section{Title of Appendix A}

	Text of Appendix A is Here

	\section{Title of Appendix B}

	Text of Appendix B is Here
	
	\section{Title of Appendix C}
	
	Text of Appendix C is Here.

\end{document}