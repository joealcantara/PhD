%%%%%%%%%%%%%%%%%%%%%%%%%%%%%%%%%%%%%%%%%%%%%%%%%%%%%%%%%%%%%%%%%%%%%%%%%%%%%%%%
%% Document Class
\documentclass[a4paper]{book}
%%%%%%%%%%%%%%%%%%%%%%%%%%%%%%%%%%%%%%%%%%%%%%%%%%%%%%%%%%%%%%%%%%%%%%%%%%%%%%%%
\begin {document}
\section {Things to do}
\begin {itemize}
    \item Learn how to use AWS or Google instances to run deep learning models
\end {itemize}
\section {Ideas to explore}
What if you extract sentences? So you have x list of sentences by controls as data. Each observation is 1 sentence. Each sentence the model reads adds to the probability AD or Not, MCI or Not. \newline
Can you hold meaning / context from previous sentences for repetition? Is this where LSTM's come in? \newline
Me collecting data? What does this look like? Then we can produce probability and confidence? This can then be tackled longditunially. \newline
Does this depend on language too much? Is there sufficient research to support acoustic analysis?
Experiment using this? In theory we migth be able to spot cognitive decline \newline
Read Berisha and Liss \newline
Can I write a paper comparing Reagan, Bush and Trump (use both original Berisha and Liss model and new deep learning model?)
If deficiencies in language are idiosyncratic, how can we generalise? How can we track changes over time?


\section {Bits of writing to go in QR, or Thesis}
\subsection {Potential Rationale for using deep learning over traditional machine learning methods.}
The recent resurgence in the use of neural networks and deep learning is because of it's significantly improved performance on many problems and it's ability to scale from small to large datasets. Another key benefit of deep learning is it's ability to automate the process of feature engineering. Feature engineering, with this particular problem is an interesting debate as there are currently numerous ways in which to try to generate features.

\subsection {On the decline of those with dementia} Whilst there are a wide number of factors that are involved in language generation, and therefore there will be an expected amount of variability between subjects. There is also an observation that as people get older, our language skills do decline. The consensus that this decline is typically exacerbated by the presence of dementia. Given this statement, it seems logical to conclude that one of the key variables that distinguishes language decline in healthy individuals vs those with dementia is the rate of change in which the decline occurs.
\end {document}