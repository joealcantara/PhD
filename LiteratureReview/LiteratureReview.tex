%%%%%%%%%%%%%%%%%%%%%%%%%%%%%%%%%%%%%%%%%%%%%%%%%%%%%%%%%%%%%%%%%%%%%%%%%%%%%%%%
%% Document Class
\documentclass[a4paper]{article}

%% Packages
\input{settings/packages}

%% Page Settings
\input{settings/page}

%% Own Commands
\input{settings/macros}

\title{Literature Review}
\date{2018-04-26}
\author{Jomar Alcantara}

%%%%%%%%%%%%%%%%%%%%%%%%%%%%%%%%%%%%%%%%%%%%%%%%%%%%%%%%%%%%%%%%%%%%%%%%%%%%%%%%
\begin {document}
\maketitle
\newpage
\tableofcontents
\newpage
\section{Introduction}
\par
Alzheimer's Disease (AD) is a neurodegenerative disease in which the brain develops neurofibrillary tangles and neuritic plaques along with the loss of cortical neurons and synapses. The hallmark clinical symptoms are cognitive deficits such as problems with episodic and semantic memory, organising and planning, difficulties with language and visuospatial deficits\cite{AmericanPsychiatricAssociation2013}. In addition, these symptoms are often accompanied by emotional difficulties such as depression and irritability and behavioural difficulties. Whilst there are several variants of Dementia, AD remains the most common type of Dementia and will be the primary focus of this review. \newline
\par
According to a recent report commissioned by the Alzheimer's Society in 2015, they estimate the prevalence of AD in the UK at approximately 815,000 people. This represents 1 in 14 of the population aged 65 or over and 1 in 79 in the general population. They estimate an annual healthcare spend on £4.3 billion of which approximately £85 million is spend solely on diagnosis and that the total impact of AD (excluding the costs associated with early onset dementia) is £26.3 billion annually. Globally, this picture is a lot bleaker. A recent report suggests that in 2015 there were 46 million people with a diagnosis of AD and that number is expected to hit 131.5 million by 2050 \cite{Prince2015}. The report also states that the worldwide cost of AD in 2018 is estimated to be in the region of one trillon US dollars. \newline
\par
Current thinking suggests that the cognitive deficits associated with AD often begin before the clinical symptoms of the disease become apparent. Researchers propose that neurofibrillary tangles and other associated physiological effects of AD develop over time until a threshold is reached and clinical symptoms become apparent \cite{Nestor2006}. If this is the case, it should be possible to detect subtle cognitive changes in language and memory before a clinical diagnosis can be formed. One of the most common ways in which clinicians make an early diagnosis of cognitive impairment is through the use of the Mini Mental State Examination \cite{Folstein1975}.

 Given that language is less intrusive to test and requires a lot of the cognitive processes that may be impacted by AD, a lot of research has focused on the decline in the use of language in those with AD.  \newline
\par
One of the most famous pieces of research was by Berisha and Liss (2015) \cite{Berisha2015} who examined speeches and public interviews of former US president Ronald Reagan. They found that Reagan's speeches towards the end of his presidency suffered from difficulties in word-finding, inappropriate phrases and uncorrected sentences which are hallmarks of language deterioration associated with Alzheimer's Disease. It turned out later to be the case that he had Alzheimer's Disease. Another classical study by Snowdon et al (1996) \cite{Snowdon1996} looked at whether linguistic ability in early life was associated with cognitive function and AD in later life. They found that idea density (defined as the number of expressed propositions divided by the number of words) was a key predictor in predicting whether nuns would go on to develop AD in later life. They found that those who would go on to develop AD all had low idea density in early life and they found no AD present in those with high idea density in early life. As we can see, just with these two pieces of research the range of language deficits in those who suffer with AD are wide and varied and differ as the disease develops. \newline
\par
According to the DSM 5 \cite{AmericanPsychiatricAssociation2013}, those with Mild dementia suffer from noticeable word finding difficulty. They may substitute general terms for more specific terms and may avoid the use of specific names of acquaintances. There may be grammatical errors involving subtle omission or incorrect use of articles, prepositions, auxiliary verbs, etc. Those who have progressed from Mild to Major depression also have difficulties with expressive or receptive language. They will often use general-use phrases such as "that thing" and "you know what I mean" and prefers general pronouns rather than names. With severe impairment, sufferers may not even recall names of closer friends and family. Idiosyncratic word usage, grammatical errors, and spontaneity of output and economy of utterances occur. Echolalia (meaningless repetition of another person's spoken words) and automatic speech typically precede mutism. With the wide range of deficits someone with AD can suffer, it makes sense to try to categorise these deficits in some way.\newline
\par
Emery \cite{Emery2000} completed a literature review looking at all the potential language deficits that could exist in those with AD and / or MCI. She divided these deficits into four levels of language: Phonology, Morphology, Syntax and Semantics. She proposed that language and the processes involved in language are hierarchical in nature and that language moves from simple units of construction (Phonology and Morphology), and build layers of complexity and sophistication (Syntax and Semantics). She found that people with AD generally had intact Phonology and Morphology but more impaired Syntax and Semantics. She stated that the language forms we learn last are the first to deteriorate. We generally learn language in small simple units initially and build syntax and complexity as we are more comfortable with language.\newline
\par
It is clear from both the clinical diagnostic criteria and supporting research that language is impacted in those with AD. However, one of the costs of analysing language is that there is a huge burden on trained practitioners, be it clinical psychologists, audio transcribers and text encoders to facilitate the process of collecting data and analysis. The field of machine learning and natural language processing has been suggested as a way to improve the accuracy and lessen the human cost of this research as well as provide new insights into the difficulties that AD suffer in terms of language decline \cite{Boschi2017}. \newline

\subsection{Aims and Methodology}
\par 
The purpose of this review is to seek to understand what techniques for assessing language have been used within the field of cognitive and clinical psychology. We then go on to look at what techniques have been developed in the field of machine learning (ML), deep learning (DL) and natural language processing (NLP) that might enable the automated analysis of language easier. Finally, we look at some studies which have already looked at the intersection of these two domains. 
 
when considering the application of Machine Learning and Natural Language processing to aid the diagnosis of Mild Cognitive Impairment (MCI) and AD. A search of the literature was conducted using ProQuest (PsychArticles), SCOPUS, Web of Science. The following results were found (Table 1). All papers were then reviewed for relevance by reading the abstract and full text where appropriate and a shortlist was compiled. An additional search through references of shortlisted papers was also conducted and any papers who upon further review appeared relevant were added to the shortlist. Papers were included where researchers used machine learning to classify participants as MCI, AD or Healthy using language. We excluded any papers that focused on other forms of dementia or cognitive impairment, as well as any papers in which the language being analysed was not English. This resulted in 17 journal articles and/or conference papers which form this review.

\begin{table}
	\begin{tabular}{ | c | c | c | p{1cm} |}
		\hline
		Database & Number of Results & Search Terms  \\ \hline
		ProQuest(PsychArticles) & 1484 Results & Language AND Decline AND Dementia \\ \hline
		ProQuest(PsychArticles) & 486 Results  & Language AND Decline AND Dementia AND Speech \\ \hline
		ProQuest(PsychArticles) & 159 Results & Machine Learning AND Dementia AND Language \\ \hline
		Web of Science & 1207 Results  & Language AND Decline AND Dementia   \\ \hline
		Web of Science & 151 Results  & Language AND Decline AND Dementia AND Speech  \\ \hline
		Web of Science & 34 Results & Machine Learning AND Dementia AND Language \\ \hline
		Scopus & 791 Results & Language AND Decline AND Dementia  \\ \hline
		Scopus & 91 Results & Language AND Decline AND Dementia AND Speech   \\ \hline
		Scopus & 29 Results & Machine Learning AND Dementia AND Langauge \\ \hline
	\end{tabular}
	\caption{\label{tab:table-name}Search Terms and Number of Results.}
\end{table}



\section {Types of Language Assessment}
One of the key debates when looking at how to analyse language is the type of task provided to elicit language production in participants. In the literature researchers have primarily focused on Picture Description tasks but have also suggested other ways in which we might collect data. \newline
\par
\subsection{Picture Description Tasks}
One of the most commonly used tasks to measure language is the Picture Description task. An example of this is part of the Boston Diagnostic Aphasia Examination (BDAE), called the Boston Cookie Theft picture description task \cite{Kaplan2010}. In this task participants are asked to describe a picture presented to them in as much detail as possible. The picture itself depicts a familiar domestic scene and would not require participants to use any vocabulary beyond that learned in childhood. It was originally designed to assess Aphasia, but has shown itself to be useful in the assessment of language for the purposes of diagnosis of MCI and AD as well \cite{Giles1996}\newline
\par
The picture description task does a fine job of eliciting descriptive language, however because of the limited content has limited use. The task in itself just a descriptive task, and therefore elicits a certain type of language. There is some disagreement as to the benefits of this using this methodology. This task is reported as being useful to lexico-semantic disorders \cite{Boschi2017, Sajjadi2012} as the language being generated is primarly nouns and deixis (words to identify items and words to put those items into context). However, Ash \cite{Ash2012}felt that there was no difference in using this task vs Story Narration (described below). In explaining the differences, it is worth noting that these researchers were using differing variables and this could explain their different perspectives. \newline
\par
In terms of Machine Learning research in this area, a number of researchers have used transcripts based on picture description tasks \cite{Zimmerer2016, Orimaye2017, Mueller2018a, Fraser2015} and have successfully extracted linguistic features that could differentiate between AD and controls.\newline
\par
\subsection{Narrative description task}
The story narration task is designed to study a participant's ability to describe and elaborate on a story which is depicted using a series of pictures. The stories depicted are usually based on children's books or famous stories such as Cinderella. \cite{Fraser2014} This task requires ordering the story in a structured and coherent framework. It also requires comprehension and understanding of the stories characters and the events depicted, as well as an awareness of a character's goals and internal responses to given events. This task is particularly useful, as the procedure reduces the demands on memory and is therefore able rule out memory as a confounding variable for any results observed. As noted above, Ash \cite{Ash2012} felt that this task was interchangable with the Picture Description task and because this task requires elaboration rather than simple description, is a sturdier test of lexical and semantic abilities as well as syntactic complexity. \cite{DeLira2011} \newline
\par
Given the relative strengths of the Narrative description task vs Picture description task, there are few pieces of research research that have used Machine Learning to analyse features from Narrative picture tasks \cite{Fraser2014}. This could be down to the availability of data and the absence of any meaningful sets of transcripts of participants performing this task. However, this could be an interesting direction to take research in the future to see if features generated from this task could be used to predict MCI or AD.  \newline
\par
\subsection{Interviews}
Interviews can also be used to elicit language, the idea of employing questions to guide a conversation between speakers. There are three types of interviews: unstructured, structured and semi-structured. Structured interviews tend to produce very limited speech and therefore has never been used in this area \cite{Boschi2017}. Unstructured interviews are open ended and generally do not conform to any particular pattern. They use generic themes such as family or hobbies to guide the conversation. Whilst this is the most ecologically valid form of conversation and therefore language generation, it's unstructured nature means that the protocol cannot be consistent and therefore reproduced. Semi-structured interviews are therefore preferred other forms of interview as a middle ground. The semi structured nature of these interviews means that there is some replicability but does not constrain the participant in answering questions. \newline
\par
The analysis of interviews can be difficult to analyse as both the content can vary even between participants. It is also difficult to measure as there are no pre-defined task goals in comparison to the other two methods. Nevertheless, this is the most naturalistic setting for looking at language production and can be used to look at the syntactic and semantic parts of language generation \cite{Sajjadi2012}. There have been some attempts to use interviews to to assess language production in AD with promising results \cite{Asgari2017, Guinn2015}.\newline
\par
\section{How do we analyse language?}
Part of the reason we need to pay attention to how we ask participants to generate data is understanding how we wish to analyse the data afterwards. As discussed above, the different methodologies to collect data generate different types of language. There are two main approaches which we have looked at to analyse language, using frequencies of words and combinations of words and measures of syntax and semantics. There are other less common methods of analysing language but these are beyond the scope of this review. \newline
\par
\subsection{Using frequencies of words and non-words}
The basic premise behind using frequencies to measure changes in language is that certain types of words and non words are more prominent in those suffering with AD than healthy controls. For example, Guinn (2012, 2015) \cite{Guinn2012, Guinn2015} found that 'Go-ahead utterances' - instances in diaglogue in which a speaker provides responsees do not add anything in a conversation beyond a minimal response, were significantly more frequent in those with AD than healthy controls..\newline 
\par
One of the first features discussed as a potential predictor of MCI or AD is the n-gram. An n-gram is a contiguous sequence of n items from a given sample of text or speech. The items can be phonemes, syllables, letters, words or base pairs according to the application. For example, given the sequence of words "to be or not to be", this extract is said to contain six 1-gram sequences (to, be, or, not, to, be), five 2-gram sequences (to be, be or, or not, not to, to be), four 3-gram sequences(to be or, be or not, or not to, not to be) and so on. This is useful as, given a large portion of text or speech, we can predict the probability of a word being close by to a given word. \newline  
\par
A number of researchers have used n-grams as features. One of the first attempts to use machine learning and natural language techniques to look was conducted by Thomas \cite{Thomas2005} who was able to successfully demonstrate the ability of machine learning algorithms to analyse n-grams as well as other features to outperform a naive rule-based classifier which always selects the modal class. Orimaye et al (2017) \cite{Orimaye2017} investigated the use of machine learning algorithms to detect differences primarily in n-gram use to distinguish between those with a diagnosis of AD and healthy controls. Their main finding supported n-grams as the most significant predictor. One of the criticisms is the use of picture description tasks and n-grams. Because the language generated by this task is content specific the n-grams generated are only specific to the task given and cannot be generalised.  \newline
\par
Asgari, Kaye and Dodge (2017) \cite{Asgari2017} used another form of word frequency measurement. Using recordings of unstructured conversations (with standardized preselected topics across subjects) between interviewers and interviewees they grouped spoken words using Linguistic Inquiry and Word Count (LIWC) which is a technique used to categorize words into features such as negative and positive words \cite{Pennebaker2015}. They were able to successfully used machine learning algorithms to distinguish between these two groups with an accuracy of 84\%. \newline
\par
\subsection{Measures of Syntactic and Semantic Complexity}
Another approach to linguistic analysis in this field is the idea of measuing syntactic and semantic complexity. According to Emery (2000) \cite{Emery2000} in which she states that Semantic and Syntactic skills deteriorate first in people with MCI and AD. If this is true, then psychological measures of semantic and syntactic skills should be able to pick up signs of deterioration and act as markers for possible MCI and AD. An example of these measures is the concept of idea density. Formally, idea density is defined as the average number of propositions per sentence \cite{Kintsch1973} and this was used to successfully differentiate between people who would later go on to develop AD \cite{Snowdon1996}. Type to Token Ratio, Brunet's Index and Honore's Statistic are two more examples of validated measures which are used to measure the lexical diversity of a given piece of text. This has also shown to be effective in differentiating between MCI, AD and Controls, with those with language impairments \cite{Bucks2000} and this has carried through in research involving machine learning \cite{Wang2016, Thomas2005}.  \newline
\par
Fraser, Meltzer and Rudzicz (2015) \cite{Fraser2015} looked at connected speech using the DementiaBank corpus. They found that there were four factors which informed the classification of participants as either healthy or AD. These four factors were semantic impairment, acoustic abnormality, syntactic impairment and information impairment and were based on existing measures of semantic and syntactic complexity. Zimmerer (2016) \cite{Zimmerer2016} looked at whether language was more formulaic in those suffering from AD. He proposed that those who suffer from AD rely on formulatic sentences, for example 'Noun-Verb-Noun', and this is done to reduce language complexity. He noticed a significant difference in the use of formulaic sentences between AD and Healthy Controls. \newline
\par
It appears that both frequency based approaches and approaches that use measures of linguistic complexity have both been shown to be effective in various experimental settings. The fact that there are two vastly different perspectives to tackling this question, and even within these perspectives there are a range of various methodologies leaves the field with a sense of confusion as to the most effective way to solving the problem. \newline
\par

\section{Measures of Lexical Richness}
\subsection{Type token ratio(TTR)}
Type token ratio (TTR) is the ratio obtained by dividing the types (The total number of different words) by the tokens (the total number of words in an utterance).
\begin{equation} \label{x1}
TTR = numberOfUniqueWords / totalNumberOfWords.
\end{equation}
\subsection{Brunet's Index(W)} % Don't forget to cite the original article.
Brunet's Index (W) differentiates itself for TTR, as it is not impacted by the length of the text itself. Brunet's Index is defined by the following equation:
\begin{equation} \label{x2}
W = N^{V(-0.165)}
\end{equation}
where N is the total length of the utterance being measured and V is equal to the total vocabulary being used by the subject. Brunet's Index usually has a score of between 10 and 20, with high numbers indicating a more rich vocabulary compared to low numbers. \newline
\par 
\subsection{Honore's Statistic (R)} % Don't forget to cite the original article.
Honore's Statistic is based on the idea that vocabulary richness is implied when a speaker uses a greater amount of unique words. This is indicated by the following equation: 
\begin{equation} \label{x3}
%% Check to see if this is accurate.
R = (100 \log N) / (1 - V1/V)
\end{equation}
where v1 is equal to the number of unique words, V is the total vocabulary used and N is the total number of words in the utterance being measured.

\section{State of literature into Natural Language processing techniques}


\section{Conclusions and Future Work}
\par
Identification of language impairment is key idea in the diagnosis of AD and early diagnosis can alter the prognosis and change the management of this degenerative disorder as well as provide new opportunities to study the progression of the disease. Given the burden on the diagnosis of AD on professionals there has been a call to use technology to potentially ease this burden \cite{Boschi2017}. It has already been shown that analysis of speech and language has shown markers that pre-date the official diagnosis of dementia \cite{Snowdon1996, Berisha2015}. A significant amount of research has gone into the use of machine learning techniques to potentially look at the automated classification of participants with MCI and/or AD, however this is a new area of research and there are some gaps in our knowledge. \newline 
\par
One of the areas for research to study is a careful examination of the features that are being used to measure language deterioration. For example, Zimmerer (2016) \cite{Zimmerer2016} describes connectivity in such as a way that correlates directly with what Mueller (2018) \cite{Mueller2018a} calls Fluency. Whilst these are very nuanced measures which differ slightly in the form they take, the sheer range of measures and features being produced make it difficult to conceptualise as part of the larger problem. Some work needs to be done in producing a consistent list of measures that are validated using existing datasets and can be used for future research moving forward. \newline
\par
\subsection{Future Work}
Future research should also be directed towards developing non-intrusive ways of detecting subtle changes in language based in the home, such that any deterioration that could indicate the presence of MCI or AD and be flagged up early in the progression of any potential cogntive impairment for further review via referral. Machine learning approaches seem to be the most logical approach for achieving this aim. Further, despite the quality of datasets being used to 'backtest' these algorithms, further research should look at generating additional datasets to increase the validity of the results found so far as well as using other methods to generate data other than Picture Description tasks which we could claim are limited in scope. \newline
\par
This area of research is extremely promising in its early results and the impact of successful research would be life changing for both individuals and the health of the worlds aging population in general. \newline

\bibliographystyle{unsrt}
\bibliography{LiteratureReview}
\end {document}