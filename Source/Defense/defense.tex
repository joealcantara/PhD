\documentclass{article}
\title{Defense of decisions}
\date{2018-04-17}
\author{Jomar Alcantara}

\begin{document}
	Question - Why focus on just language rather than speech and language given you are collecting both?
	Answer - Orimaye et al, 2017 suggested that distinction of healthy controls and those with dementia could be done solely with processing of transcripts. Given the abundance of sources of data based on transcripts and the potential burden of processing two types of data means that to keep within the scope of the project and the focus on Natural Language Processing I made the decision to see whether this could be done with just the processing of language. BETTER ANSWER?
	
	Question - Why did you pick the FCSRDT as one of your test battery?
	Answer - Research from the University of Lancaster suggested that this was the most effective test for differentiating between controls, MCI and full blown AD.
	
	Question - Why Free Speech?
	Answer - Use of Speech Analyses within a Mobile Application for the Assessment of Cognitive Impairment in Elderly People by Toth suggests that Tthe fluency and free speech tasks obtain the highest accuracy rates of classifying AD vs. MD vs. MCI vs. SCI. (AD - Alzheimer's Disease, MD - Mixed Dementia, MCI - Mild Cognitive Impairment, SCI - Subjective Cognitive Impairment) Using the data, we demonstrated classification accuracy as follows: SCI vs. AD = 92\% accuracy; SCI vs. MD = 92\% accuracy; SCI vs. MCI = 86\% accuracy and MCI vs. AD = 86\%.
\end{document}