\documentclass[11pt]{article}
\begin{document}
\textbf{Response to the Qualification Report and Viva Voce feedback}
\par 
\textbf{Your feedback}
The student had a very good understanding of the literature and could summarise it well. It was a bit lacking in the machine learning area but he has the capacity for improving that and is aware of what he needs to do. His research background was good but there needed to be more focus on the research questions. These were not explored well in the QR but were much clearer in the viva. He now has some very good ideas to take forward that should lead to a Ph.D.
\par 
The main advice we would give is to be much more careful about how and what data you collect. Getting the participants to do it themselves on a daily basis by, for example, a speech diary that they record and submit, is much more practical than a battery of tests in a clinic. Plus you do not want to be asking about clinical diagnoses using, for example, PHQ9 that measures depression because this raises a lot of ethics issues that will make it harder for NHS ethics approval. Keep the data collection simple, especially as the focus of the Ph.D is about changes in language and not about the cognitive tests themselves. We do not think the latter are necessary because the focus of the Ph.D is on whether you can detect decline by differences in language use over time and not about determining the clinical underpinnings of those differences.

\section{Research Aims and Questions}
The broad objectives of my research are to explore whether the application of Natural Language Processing and Machine Learning techniques to spontaneous speech samples can improve our ability to diagnose Mild Cognitive Impairment in patients that will directly lead to Alzheimer's Disease. Another objective of my research is to provide some clarity as to which language features are most important to measure when looking for evidence of MCI and/or AD. 
\begin{itemize}
	\item \textbf{(H1)} Monitoring language samples and other measures of cognitive decline over the course of the year will be enough to detect cognitive decline in a population of those with Subjective Cognitive Decline and Mild Cognitive Impairment
	\item \textbf{(H2)} Given a number of samples of language over a given time frame, it is possible to detect language decline in participants such that you can differentiate between normal language decline, these with Alzheimers Disease and decline in those with MCI.
\end{itemize}

\section{Data Collection Methodology}


\section{Project Aims}


\end{document}