\section{Background}
Alzheimer's Disease (AD) and other forms of dementia affect a significant proportion of the geriatric population in the world today and is currently the sixth leading cause of death in the US and was named the leading killer of women in the UK. According to a recent report commissioned by the Alzheimer's Society in 2015, they estimate the prevalence of AD in the UK at approximately 815,000 people. This represents 1 in 14 of those aged 65 or over and 1 in 79 of the general population \cite{AlzheimersSociety2014}. From a financial perspective, they estimate an annual spend of £4.3 billion of which approximately £85 million is spent solely on diagnosis and that the total impact of AD (excluding the costs associated with early onset dementia) is £26.3 billion annually. Globally, this picture is a lot bleaker. Another report by Alzheimer's Disease International suggests that in 2015 there were 46 million people with a diagnosis of dementia globally and that number is expected to hit 131.5 million by 2050 \cite{Prince2015}. The report also states that the worldwide cost of AD in 2018 is estimated to be in the region of one trillon US dollars.
\par 
AD is a neurodegenerative disease in which a definitive diagnosis can only be produced at post-mortem. However, there are a number of psychological and physiological indicators that can indicate that dementia is present. From a physiological perspective, researchers have identified two proteins called beta-amyloid and tau. In a typical case, tau accumulates and eventually forms tangles inside neurons and beta-amyloid clumps into plaques which slowly builds up between neurons. At a certain point, the levels of beta-amyloid rise and trigger a more rapid spread of tau throughout the brain. Eventually, due to this and other changes, neurons lose their ability to communicate and the brain starts to shrink. This leads to some of the more psychological symptoms, those who have dementia demonstrate cognitive deficits such as problems with episodic and semantic memory, organizing and planning, difficulties with language, problems with executive function and visuospatial deficits \cite{McKhann2011}. In addition, these symptoms are often accompanied by emotional problems such as depression and behavioural difficulties. As more neurons die throughout the brain, a person with Alzheimer's gradually loses the ability to think, remember, make decisions and function independently.
\par
Despite this growing problem and an increasing understanding about how AD affects the brain there are no medications that improve the prognosis of those with AD. All the medications that are currently on the market are designed to manage symptoms. Whilst there are numerous investigational drugs in development for the treatment of AD, a larger than normal percentage (99.6\%) of these drugs fail in clinical trials (in contrast to anti-cancer drugs which have a 80\% failure rate) \cite{Cummings2014}. Researchers have proposed that a possible reason for the lack of success is that the drugs treatments are initiated too far along in the progression of the disease and thus much of the degeneration of the brain has already taken place \cite{Cummings2014}. Research has started to be more focused on AD at it's earliest stages which some literature describes as 'Mild Cognitive Impairment (MCI) due to AD'.
\par
One of the challenges of this approach is differentiating natural cognitive decline due to aging with decline due to a form of cognitive impairment or dementia. This challenge is often complicated further due to the large variation in the cognitive abilities and educational background of individuals. Albert and his team have worked to define clinical criteria which professionals can use to diagnose MCI due to AD and differentiate this from age-associated memory impairment and age-associated cognitive decline.  One of the most important observations from this piece of work is that a diagnosis of MCI requires evidence of intra-individual change and optimally requires evaluation at two or more points \cite{Albert2011}, and this is essentially to place more importance on the trajectory of a person's cognitive abilities rather than a person's cognitive ability in general. The criteria for MCI is detailed below \cite{Albert2011}.
\begin{enumerate}
	\item Concern regarding a change in cognition - A person or an informant should express concern that there is a change in cognitive ability in comparison to a previous level of performance.
	\item Impairment in one or more cognitive domains - There should be evidence of lower performance in one or more cognitive domains beyond what would be expected of a person given their age and education. 
	\item Preservation of independence in functional abilities - Whilst persons with MCI are expected to be able to maintain independence, it is common to experience mild problems in complex functional tasks which they may have been able to perform previously. This might mean that they take more time or be less efficient at completing these tasks, or it may be that may make more mistakes.
	\item Not demented - The deterioration should be mild to the point that there is no significant loss of functioning in social or occupational contexts.
\end{enumerate}
In addition to meeting the above criteria, a clinician must rule out other conditions or factors that could account for the decline in cognition with the goal to increase the likelihood that the underlying cause of this decline is dementia. 
\par
Research has shown that early diagnosis of people with AD or MCI improves sufferers quality of life and can, in some cases, slow the progress of the disease. Early diagnosis can increase the number of research opportunities for understanding the early stages of dementia and how the disease progresses so that more research can be conducted which may, in the future, lead to new treatments and other interventions.
% Do more on the state of the art in terms of the analysis of biomarkers for MCI.
\section{Statement of the Problem}
Studies that explore ways in which we can diagnose MCI / AD generally follow one of two main approaches, the analysis of biomarkers such as concentrations of amyloid-β 1-42 (Aβ42) in cerebrospinal fluid (CSF) and analysis of cognitive abilities. The first approach yields reliable results in the detection of AD in its moderate and advanced states but does not perform well during the early stages of the disease. The second approach, that of analysing the cognitive abilities of patients in memory clinics, has gained more attention in recent years due to the fact that in clinical practice it has shown promise in the early detection of AD. In addition, the analysis of the decline of cognitive abilities is comparatively inexpensive and less invasive that the first approach which commonly requires the collection of a sample of cerebro-spinal fluid which is painful for the patient involved. This has a number of benefits for countries with less developed healthcare systems, or where the burden of healthcare is more extreme. 
\par
One of the most common ways in which clinicians traditionally use the analysis of cognitive abilities to make an early diagnosis of dementia is through the use of the Mini Mental State Examination (MMSE) \cite{Folstein1975}. The MMSE is a brief questionnaire consisting of eleven questions which tests cognitive aspects of mental function and requires only 5-10 minutes to administer \cite{Folstein1975}. The MMSE is chosen due to it's effectiveness at assessing a person's cognitive mental state at a specific point in time, as well as being as sensitive to changes as a more detailed and complex assessment such as the Wechler Adult Intelligence Scale \cite{Folstein1975}. Whilst the MMSE is useful as a brief screening tool it has it's limitations. The MMSE was not specifically created to screen for dementias and therefore does not interrogate key aspects of cognitive impairment known to be affected in dementia. It also has limited value in assessing under-educated subjects and a meta-analysis on the effectiveness of the MMSE as a diagnostic tool for dementia showed that it's accuracy was low (sensitivity between 78.4\% and 85.1\% and specificity between 81.3\% and 87.8\%). As the MMSE has been shown to have low accuracy specifically in the diagnosis of dementia, it becomes necessary for professionals to employ the use of other tools or measures such as the Free Cued Selective Reminding Test (FCSRT) \cite{Grober2010} or the Montreal Cognitive Assessment (MoCA) \cite{Davis2015}.
\par 
These tests have the benefits of being much more accurate at diagnosing cognitive impairment and discriminating between dementia and other types of cognitive impairment at the cost time and training of psychological professionals such as clinical psychologists in administering these tests. However, the utility of diagnosing dementia at the point where clinical intervention is warranted is limited because at this stage both psychological and pharmacological interventions have been shown to not be effective \cite{Prince2015, Cummings2014}. In order to further our understanding of the progression of dementia it is important to detect the signs of dementia before they are clinically apparent.  
\par 
Current thinking suggests that the cognitive deficits associated with AD often begin before the clinical symptoms of the disease become apparent. Researchers propose that neurofibrillary tangles and other associated physiological effects of AD develop over time and alter cognitive function until a threshold is reached and clinical symptoms become more obvious \cite{Nestor2006}. The case of Iris Murdoch, who had a confirmed diagnosis of dementia, illustrates this theory well. Le et al \cite{Le2011} found, in their analysis of three writers and the novels they wrote, that Iris Murdoch's work declined subtly over time, but there was a steep drop off in the use of language in her last novel when, it is theorized, the symptoms of AD manifested themselves more significantly. If this theory holds true more generally, it should be possible to detect subtle cognitive changes in language and memory before a clinical diagnosis can be formed.
\par
The two main ways in which diagnosis is performed is through assessment of memory and language. Tests of memory are classically among the most accurate ways of diagnosing dementia, however these tests suffer from the same reliance on clinicians to administer these tests in a clinical setting. Language however is a lot easier to collect and can be done in more naturalistic settings. As with memory, these tests can be done over time and would be able to chart a patients language degeneration over time. Given that language is less intrusive to test and requires a lot of the cognitive processes that may be impacted by AD, a lot of research has focused on measure decline in the use of language in those with AD. There are a number of difficulties to watch out for with this approach. There are a wide number of factors that are involved in language degeneration in the elderly, and consequently there will be an expected amount of variability between subjects. The administration of such tests may induce nervousness and discomfort which may impact performance, and also repeatedly administering the same language tests for differences over time may be confounded by improved performance at tasks via practice effects. However, there is enough promise in this approach such that it could help further our understanding of the disease, it's progression and the parts of the brain affected in the early stages.
\par
McKhann and his team were tasked with developing diagnostic guidelines for dementia in such as way that the criteria were flexible enough to be used by general healthcare providers without access to specialist medical equipment.the DSM 5 \cite{McKhann2011}. They state those suffering with mild dementia generally encounter impaired language functions (speaking, reading, writing)––symptoms include: difficulty thinking of common words while speaking, hesitations ;speech, spelling, and writing errors. More specifically, they may substitute general terms for more specific terms and may avoid the use of specific names of acquaintances. There may be grammatical errors involving subtle omission or incorrect use of articles, prepositions, auxiliary verbs, etc. Those who have progressed from Mild to Major depression also have difficulties with expressive or receptive language. They will often use general-use phrases such as "that thing" and "you know what I mean" and prefers general pronouns rather than names. With severe impairment, sufferers may not even recall names of closer friends and family. Idiosyncratic word usage, grammatical errors, and spontaneity of output and economy of utterances occur. Echolalia (meaningless repetition of another person's spoken words) and automatic speech typically precede mutism. With the wide range of deficits someone with AD can suffer, it makes sense to try to categorise these deficits in some way.
\par
One of the most famous pieces of research on the topic of language decline in dementia was by Berisha and Liss who examined speeches and public interviews of former US president Ronald Reagan \cite{Berisha2015}. They found that Reagan's speeches towards the end of his presidency suffered from difficulties in word-finding, inappropriate phrases and uncorrected sentences which are hallmarks of language deterioration associated with Alzheimer's Disease. It turned out later to be the case that he had Alzheimer's Disease. Another classical study by Snowdon et al looked at whether linguistic ability in early life was associated with cognitive function and AD in later life \cite{Snowdon1996} . They found that idea density (defined as the number of expressed propositions divided by the number of words) was a key predictor in predicting whether nuns would go on to develop AD in later life. They found that those who would go on to develop AD all had low idea density in early life and they found no AD present in those with high idea density in early life. As we can see, just with these two pieces of research the range of language deficits in those who suffer with AD are extremely variable and can differ from patient to patient as the disease progresses. The consensus among researchers that this language degeneration is typically accelerated by the presence of dementia \cite{Berisha2015} and that a potential indicator of dementia is the rate of change in which the decline occurs relative to a fixed point in time rather than a comparison across a cohort of individuals. 
\par
Emery \cite{Emery2000} completed a literature review looking at all the potential language deficits that could exist in those with AD and / or MCI. She divided these deficits into four levels of language: Phonology, Morphology, Syntax and Semantics. She proposed that language and the processes involved in language are hierarchical in nature and that language moves from simple units of construction (Phonology and Morphology), and build layers of complexity and sophistication (Syntax and Semantics). She found that people with AD generally had intact Phonology and Morphology but more impaired Syntax and Semantics. She asserted that the language forms we learn last are the first to deteriorate as we generally learn language in small simple units initially and build syntax and complexity as we are more comfortable with language. However it is important to note that different variants of dementia show different deficits in terms of language productions. Regardless, dividing language in this way is useful as it allows us to detect deteriorations in different parts of language usage and therefore may provide a way of discriminating between different forms of dementia. 
\par
It is clear from both the clinical diagnostic criteria and supporting research that language is impacted in those with AD. However, whilst there is a move towards research aimed at looking more specifically at MCI we currently lack the measures that are sensitive enough to detect MCI. Given that we know language is affected before a clinical diagnosis of dementia is usually made \cite{Berisha2015, Snowdon1996, Le2011}, it makes sense to explore whether language on it's own can provide markers that may indicate a cognitive impairment that could progress to dementia. The field of machine learning and natural language processing has been suggested as a way to improve the accuracy and lessen the human cost of this research as well as provide new insights into the difficulties that AD suffer in terms of language decline \cite{Boschi2017}. I theorize that we may be able to enable an earlier diagnosis of those with MCI and AD using samples of spontaneous speech, natural language processing (NLP) and machine learning (ML).
\par
There is a large body of research that looks at the decline in language in those with MCI and AD \cite{Taler2008, Boschi2017}. However there is conflicting evidence in these studies about which declining language factors are associated of MCI and AD \cite{Taler2008, Boschi2017}. Research therefore should look at these features in more detail and a clarification of this currently disorganised picture should go some way to helping researchers further understand the disease and it's progression. Another area of focus for research of this nature is the process of collecting appropriate language samples. Whilst collecting samples of language is comparatively unintrusive, researchers recognise that these samples require a rich sample of language that potentially cannot be generated by tasks such as the picture description task. Therefore, it would be useful to explore whether spontaneous discourse such a semi-structured interview, has the ability to put pressure on both the cognitive and linguistic systems in the same way as traditional cognitive tests such that it might be able to distinguish between healthy controls, those with MCI and those with AD. As is the case with other quantitative measures of cognitive ability, contrasting individuals with MCI and Early AD is challenging due to the variation in an individual's baseline speech capacity. It is therefore prudent to measure cognitive decline over time as suggested by Albert et al \cite{Albert2011}. There is some evidence to support this approach, Berisha et al \cite{Berisha2015} demonstrated through a longitudinal language analysis of spontaneous speech that there are marked differences in this process between those who would go on to have a diagnosis of AD and a healthy control. 
\par
\section{Research Questions}
The overall objective of the research described in this thesis is to investigate the deterioration of language in people diagnosed with MCI and the early stages of AD. In order to do this I aim to answer the following research questions
\begin{enumerate}
	\item What research has been conducted in this field so far?
	\item How can we use technology to best enable the collection of language samples?
	\item Can we develop a complete data processing pipeline that processes data efficiently
	\item What can machine learning tell us about how language affects us?
\end{enumerate} 
\par 
This thesis' original contributions are an increased understanding of how language deteriorates over time in those with MCI and early AD, the exploration via a delphi method of ideas for the collection of language features via smartphone or other internet of things devices and the development and execution of a pilot study to test the feasibility of collecting high quality language samples via in-home technology. I also contribute to the research into which language features are important in the diagnosis of MCI and early AD over time. Finally, I explore the potential of one class classification as a new approach to classifying those with MCI and Early AD vs controls. The potential impact of this research in this area is immense. 
\section{Structure of thesis}
There is a lot research in this domain from a psychological perspective. In chapter 2, I look at the background and related work in the domain of language deterioration in those with MCI and AD from the psychological perspective. This includes looking at the current ways in which psychologists collect and analyse language samples and exploration of the features of language that psychologists feel are important in this area. In addition, I take a systematic look at the work of the experts in the fields of machine learning and natural language processing in the growing area of research for NLP and Machine Learning and it's application to this problem. 
\par 
The start of my research aims to answer questions around how we can use technology to aid us in collecting language samples that may facilitate the process of diagnosis. Currently, language samples are collected in memory clinics and physician's offices and is quite a time consuming process. More recently, work has been done to automate this process via a web based app called Talk2Me \cite{Komeili2019}, which seeks to use different psychological tests to assess cognitive decline through analysis of language. There are some potential flaws in this method of data collection.  In other areas, the use of smartphone technology has allowed the collection of language for the diagnosis of disorders such as Parkinson's disease and so it appears plausible to be able to use this technology and potentially other technologies such as Amazon Echo to facilitate the collection of language samples. The work described in chapter 3 therefore describes an exploration of experts opinions' on the following questions.
\begin{enumerate}
	\item How should we collect these language samples - Smartphones? IoT Devices?
	\item What tasks should we use to elicit these language samples
	\item At what frequency should we collect these language samples at
\end{enumerate}
\par 
An adapted Delphi survey methodology is used to develop consensus on the answers to these questions and the results of these are documented in this chapter. The resulting pilot study is described in Chapter 6.
\par 
Once we have a way to collect language samples, the question to consider is how we may best analyse this data. There are a number of factors that one must consider when answering the question. We look at the effectiveness of automatic speech recognition at translating speech into language, as well as the minimum quality of data that needs to be passed to any pipepline we develop. The Talk2Me app collects and then generates approximately 2000 lexico-syntactic, acoustic and semantic features {\cite Komeili2019} as part of their data collection framework. Chapter 4 discusses the challenges connected to constructing an effective data pipeline and contains two experiments.
\par 
Whilst Chapter 4 is more theoretical in it's content, Chapter 5 documents the first experiments that use this pipeline on three existing datasets. These datasets are, the Presidents Press Conferences, The Novels of three authors and the DementiaBank dataset.
\par
In chapter 6 Pilot study of the methodology developed
\par 
Finally, in chapter 7. I discuss conduct a general discussion of the results of my research. I also think about the strengths and weaknesses of my work and suggest ways in which the research have been improved. Lastly, I look at a number of areas for future work which can build on this research.

