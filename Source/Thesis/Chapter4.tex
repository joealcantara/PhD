\section{Background}
As Komeili in her paper points out, machine learning is a tool may allow earlier detection and management of change in language \cite{Komeili2019}. But as with all machine learning models, the benefits of accuracy only come when large amounts of data is used to train the model. With our target population, this problem is further exacerbated due to the relatively low incidence of people with MCI or early AD and the relatively high cost of collecting this data in this particular population. 
\par 
One potential way forward is to look not at the ill but at the healthy. This has inverts the problem of scarcity and cost of data collection, and so we now have an abundance of data that we can now use. This, in theory, would allow us to look specifically at any deviations from the healthy. This process is known as one-class classification.
\par 
 
\section{One-class classification}
In traditional machine learning, there are numerous classification algorithms that are widely used. Their utility is classifying instances of data into one (Binary classification) or more (multiclass classification) categories. This ia a fine way to look at classification in the vast majority of cases, but some difficulties do arise under certain circumstances. 