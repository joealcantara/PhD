\documentclass{article}

\title{Qualifying Report}
\date{2018-04-22}
\author{Jomar Alcantara}

\begin{document}
	\maketitle
	\newpage
	\section{Overview of Problem}
	
	Dementia has been identified as one of those fast growing difficulties facing the world. A recent report suggests that in 2015 there were 46 million people with a diagnosis of dementia and that number is expected to hit 131.5 million by 2050(CITE). The report also states that the worldwide cost of dementia in 2018 is estimated to be in the region of one trillon US dollars.
	\newline
	\par
	A lot of work has gone into language analysis based on clinician diagnosis. This takes the form of tasks such as Free Cued Selective Recall Task, the Boston Cogntive Assessment ,and the Mini Mental State Examination. All of which are quite targeted in the language required of the person being examined. A step towards the ambient detection of AD and/or MCI in population requires that we look for similar features in language used naturally as part of general conversation. Part of this work therefore, is to see whether non prescriptive tasks (that is, tasks that do not have a pre determined right answer), can elicit the same or comparable features in language such that this can be detected ambiently. The next part of this work looks at how technology, particularly the use of smart home devices such as the Amazon Echo and Google Home families of devices. 
	\newline
	\par
	The potential impact of this research is immense. Research has shown (has it?) that early diagnosis of people with AD or MCI improves people's quality of life and can slow the progress of the disease. Equally, early diagnosis can increase the number of research opportunities for understanding the early stages of dementia and how the disease progresses so that more research can be conducted which may, in the future, lead to new treatments and other interventions.
	
	\section{Literature Review}
	
	\section{Work already carried out}
	
	\subsection{April 2018}
	
	Thursday, 26th April - Workplan (First Draft Agreed)

	
	\section{Work Plan for the future}
	\subsection{Proposed Experiments}
	

	
	\section{Proposed Timetable}
	
	\subsection{May 2018}
	Wednesday, 9th May - Alexa Skills Training (All Day)
	\subsection{June 2018}
	Friday 15th June - First Draft of Literature Review to be handed in.
	\newline
	Monday 18th June - Monday 25th June - Annual Leave
	\subsection{July 2018}
	
	\subsection{August 2018}
	
	\subsection{September 2018}
	
	\subsection{October 2018}
	
	\subsection{November 2018}
	
	\subsection{December 2018}	
	
	\subsection{January 2019}
	Friday 11th January - Rough Draft of Qualifying Due
	Friday 18th January - Complete first rough pass of capturing and processing speech.
	
	\subsection{February 2019}
	Friday, 1st February - Qualifying Report Due
	
	\subsection{March 2019}
	
	\subsection{April 2019}
	Monday, 1st April - Viva Due by this point.
	\subsection{May 2019}
	
	\subsection{June 2019}
	
	\subsection{July 2019}
	
	\subsection{August 2019}
	
	\subsection{September 2019}
	
	\subsection{October 2019}
	
	\subsection{November 2019}
	
	\subsection{December 2019}
	
	\subsection{January 2020}
	
	\subsection{February 2020}
	
	\subsection{March 2020}
	
	\subsection{April 2020}	
	
	\subsection{May 2020}
	
	\subsection{June 2020}
	
	\subsection{July 2020}
	
	\subsection{August 2020}
	
	\subsection{September 2020}
	
	\subsection{October 2020}
	
	\subsection{November 2020}
	
	\subsection{December 2020}
	
	\subsection{January 2021}
	
	\subsection{February 2021}
	
	\subsection{March 2021}
	
	\section{Provision Table of Contents}
	\subsection{Chapter 1: Introduction}
	\subsection{Chapter 2: Literature Review}
	\subsection{Chapter 3: New theories, Description of experiments}
	\subsection{Chapter 4: Results}
	\subsection{Chapter 5: Discussion}
	\subsection{Chapter 6: Conclusion}
\end{document}